\documentclass[a4paper,11pt]{article}
\usepackage[utf8]{inputenc}
\usepackage[french]{babel} 
\usepackage[T1]{fontenc} 
\usepackage{textcomp}
\usepackage{amsmath,amssymb}
\usepackage{mathrsfs}
\usepackage{stmaryrd}
\usepackage{graphicx}
\usepackage[titlepage,fancysections]{polytechnique}
\graphicspath{{Image/}}

\title{Topological persistence}
\author{Antoine GROSNIT et Yassin Hamaoui}
\subtitle{TD5 - INF556}
\date{Octobre 2018}

\begin{document}
\maketitle
\section{Structure du code}

\begin{itemize}
\item La méthode $buildMatrix$ permet d'obtenir une représentation matricielle de la fonction $boundary$ appliquée à chaque simplex de la filtration. Cette matrice est représentée par une $HashMap<Integer, Set<Integer>>$, la clé correspondant à l'identifiant de chaque simplex de la filtration, les valeurs étant le set des indices $i$ pour lesquels l'élément $B_{ij} \neq 0$ où $j$ est la valeur de la clé. B est ainsi une sparse matrix dans la mesure où l'on ne stocke que les éléments non nuls de la matrice.

La table d'association simplex-identifiant, $simplToInd$,  est stockée dans une $HashMap<Set<Integer>, Integer>$ où chaque clé correspond au set des entiers correspondant aux vertex du simplex auquel il associe une valeur entière.

La construction de la matrice fait appel à la méthode $getBoundaries$ de la classe Simplex, qui prend en paramètre  $simplToInd$ et qui renvoie la liste des identifiants des simplex obtenus en appliquant la fonction boundary.

Complexité de $buildMatrix$ : on considère chaque simplex de la filtration et on calcule pour chacun sa frontière, or on a l'inégalité $d \leq log (n)$ où $d$ est la dimension du simplex et $n$ est le nombre d'éléments de la filtration. On a donc une complexité en $O(nlog(n))$\\

\item La méthode $reduceMatrix$ implémente l'algorithme vu en cours, avec une sparse matrix. On a donc, pour chaque colonne, une recherche du pivot potentiel (en $O(n)$), et on simplifie la colonne si une colonne précédemment considérée contenait le même pivot (on opère cette simplification en $O(n)$). On réitère cette opération (au plus $n$ fois) jusqu'à ce que l'on obtienne un nouveau pivot ou que la colonne soit nulle.

La complexité de $reduceMatrix$ est donc bien en $O(n^3)$.\\

\item La méthode $buildBarcode$ itère sur chacune des colonnes : si la colonne $j$ contient un pivot (ligne $i$), on ajoute la barre [$i$,$j$), sinon, on regarde si la ligne $j$ contient un pivot et on ajoute la barre [$i$,$\inf$).

La complexité de $buildBarcode$ est donc de $O(n)$.

\end{itemize}









\includegraphics[width=15cm]{"Fig1"}\\

	


\end{document}
